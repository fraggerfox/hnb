\documentclass{article}
\usepackage[T1]{fontenc}
\usepackage[latin1]{inputenc}
\begin{document}

\begin{itemize}
	\item HIERARCHICAL NOTEBOOK
	\item  
	\item hierarchical notebook (hnb) is program to organize, many kinds
	\item of data in one place, for example adresses, todo lists, ideas,
	\item book "reviews", brainstorming, organizing a speech, making a 
	\item structured packing list random notes, and probably many more I 
	\item haven't thought of yet..
	  \begin{itemize}
		\item 
	  \end{itemize}

	\item  
	\item HOW TO GET hnb?
	  \begin{itemize}
		\item 
		\item new releases of hnb will appear on the
		\item sourceforge project page
		  \begin{itemize}
			\item http://sourceforge.net/projects/hnb
		  \end{itemize}

		\item 
	  \end{itemize}

	\item INSTALLING hnb
	  \begin{itemize}
		\item 
		\item if you've got a source release you must first compile
		\item hnb,.. see further down in this file for instructions.
		\item 
		\item Windows
		  \begin{itemize}
			\item the windows binary distribution, is a .zip file
			\item containing the executable, this readme file, 
			\item an sample database, and the licence for hnb's use.
			\item 
			\item the binary is standalone, so you can just
			\item copy it wherever you'd like it to reside.
			\item 
		  \end{itemize}

		\item Linux
		  \begin{itemize}
			\item the linux binary distribution is a .tar.gz file
			\item containing the executable, this readme file, a
			\item an sample database, and the licence for hnb's use.
			\item 
			\item the binary is dynamically linked against ncurses
			\item just place it anywere in your path.. like
			\item /usr/bin, /usr/local/bin or /home/user/bin
			\item 
		  \end{itemize}

	  \end{itemize}

	\item COMPILING hnb
	  \begin{itemize}
		\item 
		\item I've only tried to compile hnb using linux, the only dependencies are
		\item ncurses so it should compile on most any unix by just typing make.
		\item 
		\item I cross compile the win32 binary using mingw32, the Makefile is
		\item probably a mess by most standards, but it works for me (tm), and
		\item will be replaced by something more appropriate at some other time.
		\item 
	  \end{itemize}

	\item USAGE DOCUMENTATION
	  \begin{itemize}
		\item 
		\item hopefully hnb is quite selfdocumenting, but the file sample.hnb
		\item contains some explaination of concepts and ideas behind hnb
		\item invoke hnb like this to look at the file.
		\item 
		\item hnb sample.hnb
		\item 
	  \end{itemize}

	\item TODO
	  \begin{itemize}
		\item moving
		  \begin{itemize}
			\item make the selected node, be dragged instead of warped
		  \end{itemize}

		\item buffer limits
		  \begin{itemize}
			\item is now increased to 2048, but is it enough?
		  \end{itemize}

		\item different collapse modes?
		\item gtk 2.0
		  \begin{itemize}
			\item there is a new interface to bind things so I can use my own codestructue :))
		  \end{itemize}

		\item curses interface
		  \begin{itemize}
			\item in editing mode, use reverse video as cursor instead of underscorehack
			\item copy node and substree
		  \end{itemize}

		\item todo
		  \begin{itemize}
			\item recursively fix parents
			\item at tue moment this is implemented on all nodes,, todo nodes as well as ordinary nodes,.. should only todo nodes be considered?
		  \end{itemize}

	  \end{itemize}

	\item CHANGELOG
	  \begin{itemize}
		\item 0.3b
		  \begin{itemize}
			\item added registry functions, and a small sample programed called reg
			\item small bugfix for node recurse()  (go right, and to top)
		  \end{itemize}

		\item 0.3 15.oct 2000
		  \begin{itemize}
			\item fixed registry
			\item implemented node2path
			\item first rendition of cgi-bin browser
			\item wordwrap
			\item resize components when terminal is resized
			\item changed display of help at the bottom
			\item various interface cleanups,.. (ingoring keys etc..)
			\item pageup/pagedown implemented (jumps a predefined 10 nodes by default)
			\item changed menu/quit interface,.. there is one big menu now,.
			\item which contains most of the thinkable options
			\item it is activated by esc/f1/f10/ctrl+k
		  \end{itemize}

		\item 0.4 march 2001
		  \begin{itemize}
			\item fixed tab'ing into children of empty nodes bug
			\item fixed first node bug of wordwrap
			\item made wordwrap wrap at end instead of end-indent
		  \end{itemize}

		\item 0.5
		  \begin{itemize}
			\item help item's update
			\item added insert capability to insert a node below the current
		  \end{itemize}

		\item 0.5a
		  \begin{itemize}
			\item the guadec hacked fixes on strics computer
			\item 
			\item added search functionality, still not added real keyboard bindings, nor input of needle for the search, but the callbacks are in place
			\item fixed the display of confirm mode in curses ui.
			\item implemented reparenting
			\item cleaned up node remove
		  \end{itemize}

		\item 0.6b 11 april 2001
		  \begin{itemize}
			\item something went wrong in the versioning,.. but what the heck..
			\item  
			\item added load,save and display of todo in standard file format
			\item should be foolproof, and still keeps the file human readable
			\item when adding a child by navigating,.. the parents node were reset to 0 for no apparent reason.
			\item changed to bz2 format on distribution file
		  \end{itemize}

		\item 0.6c
		  \begin{itemize}
			\item rearranged input structures
			\item added todo checkboxes
			\item added error mode to display modes
			\item rewamped the helpline system
		  \end{itemize}

		\item 0.6d
		  \begin{itemize}
			\item fixed resizing
			\item added input of search term in find
			\item made search case insensitive
			\item refined search control interface
			\item made todocheckboxes become autocheckin for one level(parent),.. but not recursively yet. but it seeds the idea of what is to come.
		  \end{itemize}

		\item 0.7 13 april 2001
		\item 0.7a
		  \begin{itemize}
			\item fixed the todo structural thing
		  \end{itemize}

	  \end{itemize}

	\item 
	\item pippin@users.sourceforge.net
\end{itemize}

\end{document}